% Glossary file
% 
% Before \begin{document}:
% 
%   \include[automake=immediate]{glossaries-extra.tex}
%   \makeindex
%   \loadglsentries{tex-include/glossary.tex}
% 
% In document:
% 
%   ... proves that, for an \gls{asr.utt} ...
% 
% 
% End of document:
% 
%   \printglossary

% Automatic Speech Recognition (asr)
\newglossaryentry{asr.delta}{%
  name={delta (ASR)},%
  symbol={\ensuremath{\Delta}},%
  text={delta},%
  description={%
    Features extracted from an existing sequence of feature vectors
    $\mathbf{x}$ by taking a rolling average over the sequence dimension.
    Double-deltas are computed by taking the rolling average of deltas.%
  }%
}%
\newglossaryentry{asr.ddelta}{
  name={double delta (ASR)},%
  symbol={\ensuremath{\Delta\Delta}},%
  text={double delta},%
  description={},%
  see={asr.delta},%
}%
\newglossaryentry{asr.trans}{%
  name={transcription},%
  description={%
    A token sequence corresponding to what was said in an utterance. Usually
    words, but can be sub-words, characters, or phones.%
  },%
  seealso={asr.utt,speech.phone},%
}%
\newglossaryentry{asr.utt}{%
  name={utterance},%
  description={%
    An interval of speech, usually delineated by some acoustic or linguistic
    marker such as a pause, the end of a phrase, or sentence.%
  },%
}%

% Group theory (grp)
\newglossaryentry{grp.abelian}{%
  name={abelian group},%
  description={A group whose multiplicative operation is also commutative.},%
  seealso={grp.grp},%
}%
\newglossaryentry{grp.comp}{%
  name={compact group},%
  description={A topological group whose topology is compact},%
  seealso={grp.top,top.comp},%
}%
\newglossaryentry{grp.cos}{%
  name={coset},%
  description={},%
  see={grp.qspace},%
}%
\newglossaryentry{grp.grp}{%
  name={group},%
  symbol={\ensuremath{\mathcal{G}}},%
  description={%
    A set $\mathcal{G}$ and a closed, binary operation on said set, which has
    an identity, an inverse, and is associative.%
  },%
}%
\newglossaryentry{grp.homo}{%
  name={group homomorphism},%
  description={%
    A map $\phi$ from one group $g \in \mathcal{G}$ to another which commutes
    with group multiplication: $\phi(gg') = \phi(g)\phi(g')$.%
  },%
  seealso={grp.grp},%
}%
\newglossaryentry{grp.hspace}{%
  name={homogeneous space},%
  description={%
    A set $\mathcal{X}$ whose elements a group $\mathcal{G}$ acts on
    transitively. That is, $\forall x,x' \in \mathcal{X},\exists g \in
    \mathcal{G}: x = gx'$.%
  },%
  seealso={grp.grp}%
}%
\newglossaryentry{grp.lcoset}{%
  name={left coset},%
  description={},%
  alias={grp.cos},%
}
\newglossaryentry{grp.qspace}{%
  name={quotient space (group)},%
  text={quotient space},%
  symbol={\ensuremath{\mathcal{G}/\mathcal{H}}},%
  description={%
    For group $\mathcal{G}$ and subgroup $\mathcal{H}$, the left quotient
    space, denoted $\mathcal{G}/\mathcal{H}$, contains equivalence classes $[g]
    = \{g : gh \in \mathcal{H}\}$ (also called cosets and denoted $gH$) under
    the equivalence relation $g \sim g' \iff g^{-1}g' \in \mathcal{H}$.%
  },%
  seealso={math.qspace,grp.rqspace}%
}%
\newglossaryentry{grp.rcos}{%
  name={right coset},%
  description={},%
  see={grp.rqspace},%
}%
\newglossaryentry{grp.rqspace}{
  name={right quotient space},%
  symbol={\ensuremath{\mathcal{H}\setminus\mathcal{G}}},%
  description={%
    Like the (left) quotient space, but with elements (right cosets) defined by
    actions on elements of $\mathcal{H}$ on the right (and $g \sim g' \iff
    gg'^{-1} \in \mathcal{H}$).%
  },%
  seealso={grp.qspace},%
}%
\newglossaryentry{grp.semi}{%
  name={semigroup},%
  description={A set together with a closed, associative binary operation.},%
}%
\newglossaryentry{grp.top}{%
  name={topological group},%
  description={
    A group on a topological space wherein group multiplication is continuous.%
  },%
  seealso={grp.grp,top.cont},%
}%

% Math (math)
\newglossaryentry{math.bspace}{%
  name={Banach space},%
  description={%
    A normed vector space whose induced metric makes it a complete metric space
    as well.%
  },%
  seealso={math.norm,math.cmspace},%
}%
\newglossaryentry{math.cauch}{%
  name={Cauchy sequence},%
  description={%
    A sequence of values from a measure space $(\mathcal{X}, d)$, $x_1, x_2,
    \ldots \in \mathcal{X}$, such that, for any $\epsilon \in \mathbb{R}_{>0}$,
    there exists some $N \in \mathbb{N}$ such that, for all $x_n,x_{n'}$ where
    $n, n' \geq N$, $d(x_n, x_{n'}) \leq \epsilon$.%
  },%
  seealso={math.met},%
}%
\newglossaryentry{math.cconj}{%
  name={complex conjugate},%
  symbol={\ensuremath{\overline{\cdot}}},%
  description={%
    The complex conjugate of complex number $\mathbb{C} \ni c = a + bi$, where
    $a,b \in \mathbb{R}$ and $i$ is the imaginary unit, is $\overline{c} = a -
    bi$. The complex conjugate of function $f : \mathbb{F} \to \mathbb{C}$ is a
    operator $f \mapsto \overline{f}$ \emph{s.t.} $\overline{f}(x) =
    \overline{f(x)}$. Since $\mathbb{R} \ni c = a + 0i \in \mathbb{C}$, a real
    number is its own conjugate.%
  },%
}%
\newglossaryentry{math.cmspace}{%
  name={complete metric space},%
  description={%
    A metric space whose Cauchy sequences all have limits contained in that
    space.%
  },%
  seealso={math.met,math.cauch},%
}%
\newglossaryentry{math.cint}{%
  name={closed interval},%
  symbol={\ensuremath{\left[a, b\right]}},%
  description={%
    The ordered set of all values between $x$ and $y$, including boundaries.
    $\left[a,b\right] = \{c : a \leq c \leq b\}$.%
  },%
}%
\newglossaryentry{math.dist}{%
  name={distance},%
  description={},%
  see={math.met},%
}%
\newglossaryentry{math.eclass}{%
  name={equivalence class},%
  plural={equivalence classes},%
  description={%
    A subset of the elements of a set $x,x' \in \mathcal{X}$ which are
    equivalent with respect to some equivalence relation $\sim$. The class can
    is usually denoted $[x]$, where $x \in \mathcal{X}$ is any member in the
    class (called a representative). Any member can be a representative,
    \emph{i.e.} $[x'] = [x] \iff x' \sim x$.%
  },
  symbol={\ensuremath{[x]}}%
}%
\newglossaryentry{math.edist}{%
  name={Euclidean distance},%
  description={},%
  see={math.met,math.emet},%
}%
\newglossaryentry{math.emet}{%
  name={Euclidean metric},%
  description={%
    For $\mathbf{x}, \mathbf{x}' \in \mathbb{R}^N$, the metric 
    $d(\mathbf{x}, \mathbf{x'}) = \sqrt{\sum_n (x_n - x'_n)^2}$.%
  },%
  seealso={math.met},%
}%
\newglossaryentry{math.field}{%
  name={field},%
  symbol={\ensuremath{\mathbb{F}}},%
  description={%
    A set endowed with addition and multiplication of elements (called
    scalars); their identities ($0$ and $1$); inverses (subtraction and
    division, the latter not defined for $0$); commutativity; associativity;
    and distributivity.%
  },%
}%
\newglossaryentry{math.hspace}{%
  name={Hilbert space},%
  description={%
    An inner product space whose induced norm makes it a Banach space as well.%
  },%
  seealso={math.iprod,math.bspace},%
}%
\newglossaryentry{math.ifamily}{%
  name={indexed family},%
  plural={indexed families},%
  symbol={\ensuremath{(x_y)_{y \in \mathcal{Y}}}},%
  description={%
    Ordered elements from a set $x \in \mathcal{X}$ indexed by the set $y \in
    \mathcal{Y}$.%
  },%
}%
\newglossaryentry{math.iprod}{%
  name={inner product},%
  symbol={\ensuremath{\langle\cdot,\cdot\rangle}},%
  description={%
    A binary operator over vectors $\langle\cdot,\cdot\rangle : \mathcal{X}
    \times \mathcal{X} \to \mathbb{F}$ which is: conjugate symmetric,
    $\langle\mathbf{x},\mathbf{x}'\rangle =
    \overline{\langle\mathbf{x}',\mathbf{x}\rangle}$; linear (and
    conjugate-bilinear), $\langle\alpha\mathbf{x} +
    \alpha'\mathbf{x}',\mathbf{x}''\rangle =
    \alpha\langle\mathbf{x},\mathbf{x}''\rangle +
    \alpha'\langle\mathbf{x}',\mathbf{x}''\rangle$; and positive-definite,
    $\langle\mathbf{x},\mathbf{x}\rangle \geq 0$,
    $\langle\mathbf{x},\mathbf{x}\rangle = 0 \iff \mathbf{x} = \mathbf{0}$. A
    vector space equipped with an inner product is called an inner product
    space. It is also a normed vector space by the induced norm $\|\mathbf{x}\|
    = \sqrt{\langle\mathbf{x},\mathbf{x}\rangle}$.%
  },%
  seealso={math.norm,math.cconj},
}%
\newglossaryentry{math.irange}{%
  name={inclusive range},%
  symbol={\ensuremath{x\mathord{:}y}},%
  description={%
    The ordered set of natural numbers between $x$ and $y$ inclusive.%
  }%
}%
\newglossaryentry{math.ispace}{%
  name={inner product space},%
  description={},%
  see={math.iprod},%
}%
\newglossaryentry{math.lip}{%
  name={continuous (Lipschitz)},%
  text={Lipschitz continuous},%
  description={%
    A function $f : \mathcal{X} \to \mathcal{X}'$ on metric spaces
    $(\mathcal{X}, d)$ and $(\mathcal{X}', d')$ is Lipschitz continuous (or
    just ``Lipschitz'') \emph{iff.} there exists some $C \in \mathbb{R}_{\geq
    0}$ such that, for all $x, x' \in \mathcal{X}$, $d'(f(x), f(x')) \leq C
    d(x, x')$.%
  },%
}%
\newglossaryentry{math.met}{%
  name={metric},%
  symbol={\ensuremath{d(\cdot,\cdot)}},%
  description={%
    A metric is a function $d : \mathcal{X} \times \mathcal{X} \to
    \mathbb{R}_{\geq 0}$ such that, for all $x,x',x'' \in \mathcal{X}$: $x = x'
    \iff d(x, x') = 0$; $d(x, x') = d(x', x)$; and $d(x, x'') \leq d(x, x') +
    d(x', x'')$. The pair $(\mathcal{X}, d)$ is a metric space.%
  },%
}%
\newglossaryentry{math.mspace}{%
  name={metric space},%
  description={},%
  see={math.met},%
}%
\newglossaryentry{math.norm}{%
  name={norm},%
  symbol={\ensuremath{\|\cdot\|}},%
  description={%
    A non-negative operator over vectors $\|\cdot\| : \mathcal{X} \to \R_{\geq
    0}$ such that $\|\mathbf{x}\| = 0 \iff \mathbf{x} = \mathbf{0}$,
    $\|\alpha\mathbf{x}\| = |\alpha|\|\mathbf{x}\|$, and $\|\mathbf{x} +
    \mathbf{x}'\| \leq \|\mathbf{x}\| + \|\mathbf{x}\|$. A vector space
    equipped with a norm is called a normed vector space. It is also a metric
    space by the induced metric $d(\mathbf{x},\mathbf{x}') = \|\mathbf{x} -
    \mathbf{x}'\|$.%
  },%
  seealso={math.vspace,math.mspace},%
}%
\newglossaryentry{math.oint}{%
  name={open interval},%
  symbol={\ensuremath{\left]a,b\right[}},%
  description={%
    The ordered set of all values between $a$ and $b$, excluding boundaries.
    $\left]a,b\right[ = \{c : a < c < b\}$.%
  },%
}%
\newglossaryentry{math.preim}{%
  name={pre-image},%
  symbol={\ensuremath{\cdot^{-1}}},%
  description={%
    The pre-image of a function $f : \mathcal{A} \to \mathcal{B}$ on
    $\mathcal{B}' \subseteq \mathcal{B}$ is the set $f^{-1}(\mathcal{B}') = 
    \{a \in \mathcal{B} : f(a) \in \mathcal{B}'\} \subseteq \mathcal{A}$.%
  },%
}%
\newglossaryentry{math.qspace}{%
  name={quotient space},%
  symbol={\ensuremath{\mathcal{X}/\sim}},%
  description={%
    A set $\mathcal{X}/\sim$ of all the equivalence classes $[x] \in
    \mathcal{X}/\sim$ of $\mathcal{X}$ induced by the equivalence relation
    $\sim$. The quotient space partitions $\mathcal{X}$ ($\cup [x] =
    \mathcal{X}$ and $x \not \sim x' \iff [x]\cap[x'] = \emptyset$).%
  }%
}%
\newglossaryentry{math.represent}{%
  name={representative},%
  description={},%
  see={math.eclass}%
}%
\newglossaryentry{math.scalar}{%
  name={scalar},%
  description={},%
  see={math.field},%
}%
\newglossaryentry{math.vector}{%
  name={vector},%
  description={},%
  see={math.vspace},%
}%
\newglossaryentry{math.vspace}{%
  name={vector space},%
  description={%
    A set $\mathcal{X}$ (whose members $\mathbf{x},\mathbf{x}'$ are called
    vectors) equipped with two binary operators: addition ($+ : \mathcal{X}
    \times \mathcal{X} \to \mathcal{X}$) and scalar ($\alpha,\alpha' \in
    \mathbb{F}$) multiplication ($\cdot : \mathbb{F} \times \mathcal{X} \to
    \mathcal{X}$). A vector space is an abelian group \textit{w.r.t.} addition
    with identity $\mathbb{0}$. Multiplication satisfies: distributivity
    \textit{w.r.t} the space's addition, $\alpha(\mathbf{x} + \mathbf{x}') =
    \alpha \mathbf{x} + \alpha\mathbf{x}'$; \textit{w.r.t.} the field's
    addition, $(\alpha + \alpha')\mathbf{x} = \alpha\mathbf{x} +
    \alpha'\mathbf{x}'$; is associative \textit{w.r.t.} the field's
    multiplication, $(\alpha\alpha')\mathbf{x} = \alpha (\alpha'\mathbf{x})$;
    and satisfies identities $1\mathbf{x} = \mathbf{x}$ and $0\mathbf{x} =
    \mathbf{0}$.%
  },%
  seealso={grp.abelian,math.field},%
}%

% Measure Theory (meas)
\newglossaryentry{meas.ae}{%
  name={almost everywhere},%
  symbol={\ensuremath{\cdot_{a.e.}}},%
  description={%
    A property holds almost everywhere on a measure space if and only if the
    points it does not hold is a subset of measure zero.%
  },%
  seealso={meas.meas},%
}
\newglossaryentry{meas.count}{%
  name={counting measure},%
  description={%
    The measure $\mu$ on a $\sigma$-algebra $\Sigma$ which counts the number of
    elements in the set, $\mu(\sigma) = |\sigma|$. $\mu(\sigma) = \infty$ for
    infinite $\sigma$.%
  },%
  seealso={meas.meas},%
}%
\newglossaryentry{meas.ellpspace}{%
  name={\ensuremath{\ell^p}-sequence space},%
  symbol={\ensuremath{\ell^p}},%
  sort={lp space},%
  description={%
    The space $L^p(\mathbb{N})$ equipped with the counting metric.%
  },%
  seealso={meas.lpspace},%
}%
\newglossaryentry{meas.lpspace}{%
  name={\ensuremath{L^p}-space},%
  symbol={\ensuremath{L^p(\mathcal{X})}},%
  sort={Lp space},%
  description={%
    A measure space consisting of the quotient space of functions $f :
    \mathcal{X} \to \mathbb{C}$ (with $f \sim f' \iff f(x) =_{a.e.} f'(x)$)
    equipped with $p$-norm $\|f\|^p_p = \int_{x \in \mathcal{X}} |f(x)|^p
    \mathrm{d}\mu(x)$. The space is also a Banach space. If $p = 2$, it is also
    a Hilbert space with inner product $\langle f,f'\rangle = \int_{x \in
    \mathcal{X}} f(x)\overline{f'(x)} \mathrm{d}\mu(x)$.%
  },%
  seealso={meas.meas,meas.ae,math.hspace},%
}%
\newglossaryentry{meas.mablefunc}{%
  name={measurable function},%
  description={%
    A function $f : \mathcal{X} \to \mathcal{X}'$ with measurable domain
    $(\mathcal{X}, \Sigma)$ and codomain $(\mathcal{X}', \Sigma')$ is
    measurable \emph{iff.} the pre-image of every element $\sigma' \in \Sigma'$
    of the $\sigma$-algebra of the codomain is an element in the
    $\sigma$-algebra of the domain, $f^{-1}(\sigma') \in \Sigma$.%
  },%
  seealso={meas.sigma,math.preim},%
}
\newglossaryentry{meas.mablespace}{%
  name={measurable space},%
  description={},%
  see={meas.sigma},%
}%
\newglossaryentry{meas.meas}{%
  name={measure},%
  symbol={\ensuremath{\mu}},%
  description={%
    A function $\mu$ whose domain is a $\sigma$-algebra $\Sigma$ over a set
    $\mathcal{X}$ and whose codomain is the non-negative extended reals
    $\mathbb{R}_{\geq 0} \cup \{\infty\}$. It additionally has the properties:
    $\mu(\emptyset) = 0$; and for countably-indexed collection $\sigma_i \in
    \Sigma' \subseteq \Sigma$ containing disjoint subsets, $i \neq i' \implies
    \sigma_i \cap \sigma_{i'} = \emptyset$, $\mu(\cup_i \sigma_i) = \sum_i
    \mu(\sigma_i)$. The resulting triple $(\mathcal{X}, \Sigma, \mu)$ is a
    measure space.%
  },%
  seealso={meas.sigma},%
}%
\newglossaryentry{meas.mspace}{%
  name={measure space},%
  description={},%
  see={meas.meas},%
}%
\newglossaryentry{meas.si}{%
  name={square-integrable},%
  description={%
    A function is square-integrable if it is member of $L^2(\mathbb{R})$.%
  },%
  seealso={meas.lpspace},%
}%
\newglossaryentry{meas.sigma}{%
  name={\ensuremath{\sigma}-algebra},%
  sort={sigma algebra},%
  symbol={\ensuremath{\Sigma}},%
  description={%
    A collection $\Sigma$ of subsets of a set $\mathcal{X}$ such that:
    $\emptyset, \mathcal{X} \in \Sigma$; $\forall \sigma \in \Sigma$,
    $\mathcal{X} \setminus \sigma \in \Sigma$; and for countably-indexed
    collections $\Sigma' \subseteq \Sigma$, $\bigcup_{\sigma \in \Sigma'}
    \sigma$ is closed in $\Sigma$. The resulting pair $(\mathcal{X}, \Sigma)$
    is a measurable space.%
  }%
}%
\newglossaryentry{meas.sfinite}{%
  name={\ensuremath{\sigma}-finite},%
  sort={sigma-finite},%
  description={%
    A measure space $(\mathcal{X}, \Sigma, \mu)$ is $\sigma$-finite if there
    exists a countable subset of $\Sigma$, $\Sigma'$, such that for all $\sigma
    \in \Sigma'$, $\mu(\sigma) < \infty$, and $\cup_{\sigma \in \Sigma'} \sigma
    = \mathcal{X}$.%
  },%
  seealso={meas.meas},%
}%
\newglossaryentry{meas.ss}{%
  name={square-summable},%
  description={A sequence is square-summable if it is a member of $\ell^2$.},%
  seealso={meas.ellpspace},%
}%


% Linguistics (ling)
\newglossaryentry{ling.token}{%
  name={token (linguistics)},%
  text={token},%
  description={%
    An instance of a linguistic unit, \emph{e.g.} a word written on a page or
    an uttered sound. Takes on values matching types. %
  },%
  seealso={ling.type}%
}%
\newglossaryentry{ling.type}{%
  name={type (linguistics)},%
  text={type},%
  description={%
    A class or label of linguistic unit, \emph{e.g.} a word uninstantiated.
    A type gives values to tokens. %
  },%
  seealso={ling.token},%
}%
\newglossaryentry{ling.vocab}{%
  name={vocabulary},
  plural={vocabularies},
  symbol=\ensuremath{\mathcal{V}},%
  description={%
    The set of possible types which can give value to a token. Usually finite.%
  },%
  see={ling.type,ling.token},%
}%

% Machine Learning (ml)
\newglossaryentry{ml.pretrain}{%
  name={pre-training},%
  description={%
    Additional training of a statistical model prior to whatever constitutes
    the standard training regime. In speech recognition, pre-training is an
    unsupervised training stage wherein an acoustic model is trained to predict
    masked, quantized samples or frames. In language modelling, pre-training
    involves predicting masked tokens.
  },%
}%
\newglossaryentry{ml.ulearn}{%
  name={unsupervised learning},%
  description={%
    Machine learning in which training is performed without target labels.
  },%
}%

% Modular Arithmetic (mod)
\newglossaryentry{mod.cclass}{
  name={congruence class},%
  plural={congruence classes},%
  description={},%
  alias={mod.residue},%
}%
\newglossaryentry{mod.residue}{
  name={residue},%
  description={%
    An equivalence class of integers $\ldots, a - 2n, a - n, a, a + n, a + 2n,
    \ldots$ with the same value modulo some $n$.%
  },%
  seealso={math.eclass},%
}%
\newglossaryentry{mod.rclass}{
  name={residue class},%
  plural={residue classes},%
  description={},%
  alias={mod.residue},%
}%

% NLP (nlp)
\newglossaryentry{nlp.corpus}{%
  name={corpus},%
  plural={corpora},%
  description={
    A language resource, \eg text (with or without parse trees) or
    utterances and their transcriptions.%
  },%
}

% Signal Processing (sp)
\newglossaryentry{sp.admis}{%
  name={admissibility condition (wavelet)},%
  text={admissibility condition},%
  description={},%
  see={sp.wavelet},%
}%
\newglossaryentry{sp.alias}{%
  name={aliasing},%
  description={%
    Phenomenon wherein a digitized signal's frequency response cannot
    distinguish higher frequencies from lower ones due to undersampling.%
  },%
}
\newglossaryentry{sp.bandwidth}{%
  name={bandwidth},%
  description={%
    A difference between two frequencies. Those frequencies usually demarcate
    the region of a filter's non-negligible frequency response. Commonly the
    3dB bandwidth, marked by the points at which the magnitude of the response
    first drops to half that of its apex.%
  },%
}%
\newglossaryentry{sp.highpass}{%
  name={high-pass},%
  description={%
    A filtering operation which attenuates (diminishes) a signals' low
    frequency coefficients.%
  },%
}%
\newglossaryentry{sp.lowpass}{%
  name={low-pass},%
  description={%
    A filtering operation which attenuates (diminishes) a signals' high
    frequency coefficients.%
  },%
}%
\newglossaryentry{sp.narrowband}{%
  name={narrow-band},%
  description={A signal whose bandwidth is relatively short.},%
  seealso={sp.bandwidth},%
}
\newglossaryentry{sp.spectra}{%
  name={spectrum},%
  plural={spectra},%
  symbol={\ensuremath{\widehat{\cdot}}},%
  description={%
    The eigenvalues of an eigendecomposition of a bounded linear operator.
    Usually refers to the set of frequency coefficients describing a signal
    after a Fourier Transform, though other transforms may be substituted,
    (\emph{e.g.} Laplace).%
  }%
}%
\newglossaryentry{sp.spectro}{%
  name={spectrogram},%
  description={%
    A pictoral representation of the energy of a signal over time and
    frequency.%
  },%
}%
\newglossaryentry{sp.supp}{%
  name={support},%
  symbol={\ensuremath{\mathord{supp}}},%
  description={%
    The elements of the domain of $f : X \to \mathbb{F}$ such that $f$
    evaluated on those elements is non-zero. $\operatorname{supp} f = \{x :
    f(x) \neq 0\}$.%
  },%
}%
\newglossaryentry{sp.wavelet}{%
  name={wavelet},%
  description={%
    A signal $\psi \in L^2(\mathbb{R})$ also satisfying the admissibility
    condition $\int_0^\infty \frac{|\widehat{\psi}(\omega)|^2}{\omega}
    \mathrm{d}\omega < \infty$.%
  },%
}
\newglossaryentry{sp.wideband}{%
  name={wide-band},%
  description={A signal whose bandwidth is relatively long.},%
  seealso={sp.bandwidth},%
}

% Speech Production/Perception (speech)
\newglossaryentry{speech.coart}{%
  name={coarticulation},%
  description={%
    The phenonenon in speech wherein the realization of speech sounds is
    impacted by another nearby sound.%
  },%
}%
\newglossaryentry{speech.formant}{%
  name={formant},%
  description={A high-energy band within the a speech spectrum.},%
  seealso={sp.spectra},%
}%
\newglossaryentry{speech.phone}{%
  name={phone},%
  description={A speech sound.}%
}%
\newglossaryentry{speech.pitch}{%
  name={pitch},%
  plural={pitches},%
  description={The perception of frequency.},%
}%

% Topology (top)
\newglossaryentry{top.borel}{
  name={Borel $\sigma$-algrebra},%
  symbol={\ensuremath{\mathcal{B}(\mathcal{X})}},%
  description={%
    On topological space $\mathcal{X}$ with topology $\mathscr{X}$, its Borel
    $\sigma$-algebra is the smallest $\sigma$-algebra containing $\mathcal{X}$,
    $\cap_{\{\Sigma : \mathcal{X} \subseteq \Sigma\}} \Sigma$.%
  },
  seealso={top.top,meas.sigma},%
}%
\newglossaryentry{top.comp}{%
  name={compact topology},%
  plural={compact topologies},%
  description={%
    A topology \emph{s.t.} for any open cover of $\mathcal{X}$, \emph{i.e.}
    a collection $\mathscr{X}' \subseteq \mathcal{X}$ \emph{s.t.}
    $\mathcal{X} = \cup_{\mathcal{X}' \in \mathscr{X}'} \mathcal{X}'$, there
    exists a finite subcover, $\exists \mathscr{X}'' \subseteq \mathscr{X}'$
    \emph{s.t.} $|\mathscr{X}''| \in \mathbb{N}$ and $\mathscr{X}''$ is an
    open cover of $\mathcal{X}$. $\mathcal{X}$ equipped with the compact
    topology is considered a compact space.%
  },%
  seealso={top.top},%
}%
\newglossaryentry{top.cont}{%
  name={continuous (topology)},%
  text={continuous},%
  description={%
    A map $\phi$ between two topological spaces $\mathcal{X},\mathcal{Y}$ is
    continuous \emph{iff.}, for every open set $\mathcal{Y}' \subseteq
    \mathcal{Y}$, its pre-image $\phi^{-1}(\mathcal{Y}')$ is an open set in
    $\mathcal{X}$.%
  },%
  seealso={top.top,math.preim},%
}%
\newglossaryentry{top.compspace}{%
  name={compact space},%
  description={},%
  see={top.comp},%
}%
\newglossaryentry{top.disc}{%
  name={discrete topology},%
  plural={discrete topologies},%
  description={%
    The powerset topology over $\mathcal{X}$, \emph{i.e.} $\mathcal{X}'
    \subseteq \mathcal{X} \iff \mathcal{X}' \in \mathscr{X}$.%
  }%
}%
\newglossaryentry{top.top}{%
  name={topology},%
  plural={topologies},%
  symbol={\ensuremath{\mathscr{X}}},%
  description={%
    For a given set $\mathcal{X}$, a topology on $\mathcal{X}$ is a collection
    of subsets (called open sets) of $\mathcal{X}$, $\mathscr{X}$, such that:
    $\emptyset,\mathcal{X} \in \mathscr{X}$; for any indexed collection
    $\mathcal{X}_i \in \mathscr{X}$ (countable or uncountable), $\bigcup_i
    \mathcal{X}_i \in \mathscr{X}$; and for any finite collection
    $\mathcal{X}_i \in \mathscr{X}$, $\bigcap_i \mathcal{X}_i \in \mathscr{X}$.
    When equipped with a topology, the set $\mathcal{X}$ is considered a
    topological space.%
  },%
}%
\newglossaryentry{top.topspace}{%
  name={topological space},%
  description={},%
  see={top.top},%
}%
