
% Commands and symbols
% 
% Before \begin{document}:
% 
%    
% Commands and symbols
% 
% Before \begin{document}:
% 
%    
% Commands and symbols
% 
% Before \begin{document}:
% 
%    
% Commands and symbols
% 
% Before \begin{document}:
% 
%    \include{math.tex}
% 
% In document:
% 
% \Abs{x}        % absolute value
% \alg{A}        % algebra
% \C             % complex field
% \Card{x}       % cardinality
% \Ceil{x}       % ceiling
% \Closed{x}{y}  % closed interval
% \coll{A}       % collection
% \Conj{x}       % conjugate
% \Conv{x}{y}    % convolution - x * y
% \Conv{x}{y}[t] % convolution - (x * y)[t]
% \Conv*{x}{y}   % convolution (alternate) - x (* w/ underline) y
% \Corr{x}{y}    % correlation - x star y
% \Corr{x}{y}[t] % correlation - (x star y)[t]
% \Corr*{x}{y}   % correlation (alternate) - x (star w/ underline) y
% \Dual{x}       % dual
% \Ex[y]{x}      % expectation - E_y[x]
% \Ex{x}         % expectation - E[x]
% \field{A}      % field
% \Floor{x}      % floor
% \Fourier{x}    % Fourier transform of x
% x \iff y       % if-and-only-if
% x \asto y      % almost sure convergence to
% \Inner{x}{y}   % inner product
% \Inv{x}        % inverse
% \IRange{a}     % range - 1:a
% \IRange{a}{b}  % range - a:b
% \matr{x}       % matrix
% \N             % natural semiring
% \Norm[y]{x}    % norm - ||x||_y
% \Norm{x}       % norm - ||x||
% \Open{x}{y}    % open interval
% \Pdiv{x}{y}    % partial derivative
% \Pr[y]{x}      % probability - P_y(x)
% \Pr{x}         % probability - P(x)
% \Pr{x}{}{y}    % probability - P(x||y)
% \Pr{x}{y}      % probability - P(x|y)
% \Pr{x}{y}{z}   % probability - P(x|y||z)
% \Q             % rational field
% \ring{A}       % ring
% \semi{A}       % semiring
% \set{A}        % set
% \SInt{A}       % set integral
% \supp x        % support
% \tmat{x}       % matrix transpose
% \tvec{x}       % vector transpose
% \V             % vocabulary
% \vec{x}        % vector
% \vq            % q (vector)
% \vx            % x (vector)
% \vy            % y (vector)
% \Z             % integer ring

% spellcheck-off

\usepackage{xparse}
\usepackage{bm,upgreek,mathtools,mathrsfs}

\DeclarePairedDelimiter\Innerp{\langle}{\rangle}
\DeclarePairedDelimiter\Floor{\lfloor}{\rfloor}
\DeclarePairedDelimiter\Ceil{\lceil}{\rceil}
\newcommand\Inner[2]{\ensuremath{\Innerp{#1, #2}}}
\DeclarePairedDelimiter\Abs{\lvert}{\rvert}
\DeclarePairedDelimiter\Card{\lvert}{\rvert}
\DeclarePairedDelimiter\Closedp{[}{]}
\newcommand\Closed[2]{\ensuremath{\Closedp{#1, #2}}}
\DeclarePairedDelimiter\Openp{]}{[}
\newcommand\Open[2]{\ensuremath{\Openp{#1, #2}}}
\DeclareRobustCommand\iff{\mathbin{\Longleftrightarrow}}
\newcommand\Four[1]{\ensuremath{\widehat{#1}}}
\DeclarePairedDelimiter\Normp{\|}{\|}
\DeclareMathOperator\supp{supp}

\ExplSyntaxOn

\DeclareDocumentCommand\IRange{m g}{%
  \ensuremath{\IfNoValueTF{#2}{1}{#1} \mathbin{:} \IfNoValueTF{#2}{#1}{#2}}%
}

\DeclareDocumentCommand\Conv{s m m O{} D(){}}{%
\ensuremath{%
  \tl_if_empty:nF {#4#5} {\left(}%
  {#2}%
  \mathbin{\IfBooleanTF{#1}{\underline{\ast}}{\ast}}%
  {#3}%
  \tl_if_empty:nF {#4} {\right)\!\left[#4\right]}%
  \tl_if_empty:nF {#5} {\right)\!\left(#5\right)}%
}%
}

\DeclareDocumentCommand\Corr{s m m O{} D(){}}{%
  \ensuremath{%
    \tl_if_empty:nF {#4#5} {\left(}%
    {#2}%
    \mathbin{\IfBooleanTF{#1}{\underline{\star}}{\star}}%
    {#3}%
    \tl_if_empty:nF {#4} {\right)\!\left[#4\right]}%
    \tl_if_empty:nF {#5} {\right)\!\left(#5\right)}%
  }%
}

\DeclareDocumentCommand\Lp{O{2} G{\mathbb{R}}}{%
  \ensuremath{\mathord{{\bm L}\textsuperscript{\ensuremath{#1}}}\left(#2\right)}%
}

\DeclareDocumentCommand\Norm{o m}{%
  \ensuremath{\Normp{#2}\IfNoValueF{#1}{\textsubscript{\ensuremath{#1}}}}%
}

\NewDocumentCommand\Vector{m}
 {
  \commexo_vector:n { #1 }
 }

\cs_new_protected:Npn \commexo_vector:n #1
 {
  \tl_map_inline:nn { #1 }
   {
    \commexo_vector_inner:n { ##1 }
   }
 }

\cs_new_protected:Npn \commexo_vector_inner:n #1
 {
  \tl_if_in:VnTF \g_commexo_latin_tl { #1 }
   {% we check whether the argument is a Latin letter
    \mathbf { #1 } % a Latin letter
   }
   {% if not a Latin letter, we check if it's an uppercase Greek letter
    \tl_if_in:VnTF \g_commexo_ucgreek_tl { #1 }
     {
      \bm { #1 } % a Greek uppercase letter
     }
     {% if not, we check if it's a lowercase Greek letter
      \tl_if_in:VnTF \g_commexo_lcgreek_tl { #1 }
       {
        \commexo_makeboldupright:n { #1 }
       }
       {% none of the above, just issue #1
        #1 % fall back
       }
     }
   }
 }

\cs_new_protected:Npn \commexo_makeboldupright:n #1
 {
  \bm { \use:c { up \cs_to_str:N #1 } }
 }

\tl_new:N \g_commexo_latin_tl
\tl_new:N \g_commexo_ucgreek_tl
\tl_new:N \g_commexo_lcgreek_tl
\tl_gset:Nn \g_commexo_latin_tl
 {
  ABCDEFGHIJKLMNOPQRSTUVWXYZ
  abcdefghijklmnopqrstuvwxyz
 }
\tl_gset:Nn \g_commexo_ucgreek_tl
 {
  \Gamma\Delta\Theta\Lambda\Pi\Sigma\Upsilon\Phi\Chi\Psi\Omega
 }
\tl_gset:Nn \g_commexo_lcgreek_tl
 {
  \alpha\beta\gamma\delta\epsilon\zeta\eta\theta\iota\kappa
  \lambda\mu\nu\xi\pi\rho\sigma\tau\upsilon\phi\chi\psi\omega
  \varepsilon\vartheta\varpi\varphi\varsigma\varrho
 }

\DeclareDocumentCommand\Pr{O{} m G{} G{}}{ %
      \tl_if_empty:nTF {#1}
        {\text{P}}
        {\text{P\textsubscript{\ensuremath{#1}}}}
      \ensuremath{
      \tl_if_empty:nTF {#3}
        {\tl_if_empty:nTF {#4}
          {\left(#2\right)}
          {\left(#2\middle\|#4\right)}}
        {\tl_if_empty:nTF {#4}
          {\left(#2\middle|#3\right)}
          {\left(#2\middle|#3\middle\|#4\right)}}
      }
  }

\DeclareDocumentCommand\Ex{O{} m}
  {
    \tl_if_empty:nTF {#1}
      {\mathbb{E}}
      {\mathbb{E\text{\textsubscript{\ensuremath{#1}}}}}
    \ensuremath{\left[#2\right]}
  }

\DeclareDocumentCommand\EnumSet{s m G{}}
  {
    \ensuremath{
      \IfBooleanTF{#1} {
        \tl_if_empty:nTF {#3}
          {\left(#2\right)}
          {\left(#2\middle|#3\right)}
      }
      {
        \tl_if_empty:nTF {#3}
          {\left\{#2\right\}}
          {\left\{#2\middle|#3\right\}}
      }
    }
  }

\DeclareDocumentCommand{\Powerset}{ G{} }
  {
    \ensuremath{
      \tl_if_empty:nTF {#1}
        {\mathscr{P}}
        {\mathscr{P}\left({#1}\right)}
    }
  }

\ExplSyntaxOff

\renewcommand\vec[1]{\ensuremath{\Vector{#1}}}
\newcommand\tvec[1]{\ensuremath{\Vector{#1}^\intercal}}
\newcommand\matr\vec
\newcommand\tmatr\tvec
\newcommand\field[1]{\ensuremath{\mathbb{#1}}}
\newcommand\ring\field
\newcommand\semi\ring
\newcommand\set[1]{\ensuremath{\mathcal{#1}}}
\newcommand\coll[1]{\ensuremath{\mathscr{#1}}}
\newcommand\Inv[1]{\ensuremath{{#1}^{\text{-}1}}}
\newcommand\Dual[1]{\ensuremath{{#1}\prime}}
\newcommand\alg[1]{\ensuremath{\mathfrak{#1}}}
\newcommand\setint[1]{\ensuremath{\int\limits_{#1}\mathrm{d}#1\>}}
\newcommand\Pdiv[2]{\ensuremath{\frac{\partial{#1}}{\partial{#2}}}}
\newcommand\Conj[1]{\ensuremath{\overline{#1}}}
\newcommand\asto{\ensuremath{\mathbin{\overset{a.s.}{\to}}}}

\newcommand\R{\ensuremath{\field{R}}}
\newcommand\C{\ensuremath{\field{C}}}
\newcommand\N{\ensuremath{\semi{N}}}
\newcommand\Z{\ring{Z}}
\newcommand\Q{\field{Q}}
\newcommand\V{\set{V}}

% 
% In document:
% 
% \Abs{x}        % absolute value
% \alg{A}        % algebra
% \C             % complex field
% \Card{x}       % cardinality
% \Ceil{x}       % ceiling
% \Closed{x}{y}  % closed interval
% \coll{A}       % collection
% \Conj{x}       % conjugate
% \Conv{x}{y}    % convolution - x * y
% \Conv{x}{y}[t] % convolution - (x * y)[t]
% \Conv*{x}{y}   % convolution (alternate) - x (* w/ underline) y
% \Corr{x}{y}    % correlation - x star y
% \Corr{x}{y}[t] % correlation - (x star y)[t]
% \Corr*{x}{y}   % correlation (alternate) - x (star w/ underline) y
% \Dual{x}       % dual
% \Ex[y]{x}      % expectation - E_y[x]
% \Ex{x}         % expectation - E[x]
% \field{A}      % field
% \Floor{x}      % floor
% \Fourier{x}    % Fourier transform of x
% x \iff y       % if-and-only-if
% x \asto y      % almost sure convergence to
% \Inner{x}{y}   % inner product
% \Inv{x}        % inverse
% \IRange{a}     % range - 1:a
% \IRange{a}{b}  % range - a:b
% \matr{x}       % matrix
% \N             % natural semiring
% \Norm[y]{x}    % norm - ||x||_y
% \Norm{x}       % norm - ||x||
% \Open{x}{y}    % open interval
% \Pdiv{x}{y}    % partial derivative
% \Pr[y]{x}      % probability - P_y(x)
% \Pr{x}         % probability - P(x)
% \Pr{x}{}{y}    % probability - P(x||y)
% \Pr{x}{y}      % probability - P(x|y)
% \Pr{x}{y}{z}   % probability - P(x|y||z)
% \Q             % rational field
% \ring{A}       % ring
% \semi{A}       % semiring
% \set{A}        % set
% \SInt{A}       % set integral
% \supp x        % support
% \tmat{x}       % matrix transpose
% \tvec{x}       % vector transpose
% \V             % vocabulary
% \vec{x}        % vector
% \vq            % q (vector)
% \vx            % x (vector)
% \vy            % y (vector)
% \Z             % integer ring

% spellcheck-off

\usepackage{xparse}
\usepackage{bm,upgreek,mathtools,mathrsfs}

\DeclarePairedDelimiter\Innerp{\langle}{\rangle}
\DeclarePairedDelimiter\Floor{\lfloor}{\rfloor}
\DeclarePairedDelimiter\Ceil{\lceil}{\rceil}
\newcommand\Inner[2]{\ensuremath{\Innerp{#1, #2}}}
\DeclarePairedDelimiter\Abs{\lvert}{\rvert}
\DeclarePairedDelimiter\Card{\lvert}{\rvert}
\DeclarePairedDelimiter\Closedp{[}{]}
\newcommand\Closed[2]{\ensuremath{\Closedp{#1, #2}}}
\DeclarePairedDelimiter\Openp{]}{[}
\newcommand\Open[2]{\ensuremath{\Openp{#1, #2}}}
\DeclareRobustCommand\iff{\mathbin{\Longleftrightarrow}}
\newcommand\Four[1]{\ensuremath{\widehat{#1}}}
\DeclarePairedDelimiter\Normp{\|}{\|}
\DeclareMathOperator\supp{supp}

\ExplSyntaxOn

\DeclareDocumentCommand\IRange{m g}{%
  \ensuremath{\IfNoValueTF{#2}{1}{#1} \mathbin{:} \IfNoValueTF{#2}{#1}{#2}}%
}

\DeclareDocumentCommand\Conv{s m m O{} D(){}}{%
\ensuremath{%
  \tl_if_empty:nF {#4#5} {\left(}%
  {#2}%
  \mathbin{\IfBooleanTF{#1}{\underline{\ast}}{\ast}}%
  {#3}%
  \tl_if_empty:nF {#4} {\right)\!\left[#4\right]}%
  \tl_if_empty:nF {#5} {\right)\!\left(#5\right)}%
}%
}

\DeclareDocumentCommand\Corr{s m m O{} D(){}}{%
  \ensuremath{%
    \tl_if_empty:nF {#4#5} {\left(}%
    {#2}%
    \mathbin{\IfBooleanTF{#1}{\underline{\star}}{\star}}%
    {#3}%
    \tl_if_empty:nF {#4} {\right)\!\left[#4\right]}%
    \tl_if_empty:nF {#5} {\right)\!\left(#5\right)}%
  }%
}

\DeclareDocumentCommand\Lp{O{2} G{\mathbb{R}}}{%
  \ensuremath{\mathord{{\bm L}\textsuperscript{\ensuremath{#1}}}\left(#2\right)}%
}

\DeclareDocumentCommand\Norm{o m}{%
  \ensuremath{\Normp{#2}\IfNoValueF{#1}{\textsubscript{\ensuremath{#1}}}}%
}

\NewDocumentCommand\Vector{m}
 {
  \commexo_vector:n { #1 }
 }

\cs_new_protected:Npn \commexo_vector:n #1
 {
  \tl_map_inline:nn { #1 }
   {
    \commexo_vector_inner:n { ##1 }
   }
 }

\cs_new_protected:Npn \commexo_vector_inner:n #1
 {
  \tl_if_in:VnTF \g_commexo_latin_tl { #1 }
   {% we check whether the argument is a Latin letter
    \mathbf { #1 } % a Latin letter
   }
   {% if not a Latin letter, we check if it's an uppercase Greek letter
    \tl_if_in:VnTF \g_commexo_ucgreek_tl { #1 }
     {
      \bm { #1 } % a Greek uppercase letter
     }
     {% if not, we check if it's a lowercase Greek letter
      \tl_if_in:VnTF \g_commexo_lcgreek_tl { #1 }
       {
        \commexo_makeboldupright:n { #1 }
       }
       {% none of the above, just issue #1
        #1 % fall back
       }
     }
   }
 }

\cs_new_protected:Npn \commexo_makeboldupright:n #1
 {
  \bm { \use:c { up \cs_to_str:N #1 } }
 }

\tl_new:N \g_commexo_latin_tl
\tl_new:N \g_commexo_ucgreek_tl
\tl_new:N \g_commexo_lcgreek_tl
\tl_gset:Nn \g_commexo_latin_tl
 {
  ABCDEFGHIJKLMNOPQRSTUVWXYZ
  abcdefghijklmnopqrstuvwxyz
 }
\tl_gset:Nn \g_commexo_ucgreek_tl
 {
  \Gamma\Delta\Theta\Lambda\Pi\Sigma\Upsilon\Phi\Chi\Psi\Omega
 }
\tl_gset:Nn \g_commexo_lcgreek_tl
 {
  \alpha\beta\gamma\delta\epsilon\zeta\eta\theta\iota\kappa
  \lambda\mu\nu\xi\pi\rho\sigma\tau\upsilon\phi\chi\psi\omega
  \varepsilon\vartheta\varpi\varphi\varsigma\varrho
 }

\DeclareDocumentCommand\Pr{O{} m G{} G{}}{ %
      \tl_if_empty:nTF {#1}
        {\text{P}}
        {\text{P\textsubscript{\ensuremath{#1}}}}
      \ensuremath{
      \tl_if_empty:nTF {#3}
        {\tl_if_empty:nTF {#4}
          {\left(#2\right)}
          {\left(#2\middle\|#4\right)}}
        {\tl_if_empty:nTF {#4}
          {\left(#2\middle|#3\right)}
          {\left(#2\middle|#3\middle\|#4\right)}}
      }
  }

\DeclareDocumentCommand\Ex{O{} m}
  {
    \tl_if_empty:nTF {#1}
      {\mathbb{E}}
      {\mathbb{E\text{\textsubscript{\ensuremath{#1}}}}}
    \ensuremath{\left[#2\right]}
  }

\DeclareDocumentCommand\EnumSet{s m G{}}
  {
    \ensuremath{
      \IfBooleanTF{#1} {
        \tl_if_empty:nTF {#3}
          {\left(#2\right)}
          {\left(#2\middle|#3\right)}
      }
      {
        \tl_if_empty:nTF {#3}
          {\left\{#2\right\}}
          {\left\{#2\middle|#3\right\}}
      }
    }
  }

\DeclareDocumentCommand{\Powerset}{ G{} }
  {
    \ensuremath{
      \tl_if_empty:nTF {#1}
        {\mathscr{P}}
        {\mathscr{P}\left({#1}\right)}
    }
  }

\ExplSyntaxOff

\renewcommand\vec[1]{\ensuremath{\Vector{#1}}}
\newcommand\tvec[1]{\ensuremath{\Vector{#1}^\intercal}}
\newcommand\matr\vec
\newcommand\tmatr\tvec
\newcommand\field[1]{\ensuremath{\mathbb{#1}}}
\newcommand\ring\field
\newcommand\semi\ring
\newcommand\set[1]{\ensuremath{\mathcal{#1}}}
\newcommand\coll[1]{\ensuremath{\mathscr{#1}}}
\newcommand\Inv[1]{\ensuremath{{#1}^{\text{-}1}}}
\newcommand\Dual[1]{\ensuremath{{#1}\prime}}
\newcommand\alg[1]{\ensuremath{\mathfrak{#1}}}
\newcommand\setint[1]{\ensuremath{\int\limits_{#1}\mathrm{d}#1\>}}
\newcommand\Pdiv[2]{\ensuremath{\frac{\partial{#1}}{\partial{#2}}}}
\newcommand\Conj[1]{\ensuremath{\overline{#1}}}
\newcommand\asto{\ensuremath{\mathbin{\overset{a.s.}{\to}}}}

\newcommand\R{\ensuremath{\field{R}}}
\newcommand\C{\ensuremath{\field{C}}}
\newcommand\N{\ensuremath{\semi{N}}}
\newcommand\Z{\ring{Z}}
\newcommand\Q{\field{Q}}
\newcommand\V{\set{V}}

% 
% In document:
% 
% \Abs{x}        % absolute value
% \alg{A}        % algebra
% \C             % complex field
% \Card{x}       % cardinality
% \Ceil{x}       % ceiling
% \Closed{x}{y}  % closed interval
% \coll{A}       % collection
% \Conj{x}       % conjugate
% \Conv{x}{y}    % convolution - x * y
% \Conv{x}{y}[t] % convolution - (x * y)[t]
% \Conv*{x}{y}   % convolution (alternate) - x (* w/ underline) y
% \Corr{x}{y}    % correlation - x star y
% \Corr{x}{y}[t] % correlation - (x star y)[t]
% \Corr*{x}{y}   % correlation (alternate) - x (star w/ underline) y
% \Dual{x}       % dual
% \Ex[y]{x}      % expectation - E_y[x]
% \Ex{x}         % expectation - E[x]
% \field{A}      % field
% \Floor{x}      % floor
% \Fourier{x}    % Fourier transform of x
% x \iff y       % if-and-only-if
% x \asto y      % almost sure convergence to
% \Inner{x}{y}   % inner product
% \Inv{x}        % inverse
% \IRange{a}     % range - 1:a
% \IRange{a}{b}  % range - a:b
% \matr{x}       % matrix
% \N             % natural semiring
% \Norm[y]{x}    % norm - ||x||_y
% \Norm{x}       % norm - ||x||
% \Open{x}{y}    % open interval
% \Pdiv{x}{y}    % partial derivative
% \Pr[y]{x}      % probability - P_y(x)
% \Pr{x}         % probability - P(x)
% \Pr{x}{}{y}    % probability - P(x||y)
% \Pr{x}{y}      % probability - P(x|y)
% \Pr{x}{y}{z}   % probability - P(x|y||z)
% \Q             % rational field
% \ring{A}       % ring
% \semi{A}       % semiring
% \set{A}        % set
% \SInt{A}       % set integral
% \supp x        % support
% \tmat{x}       % matrix transpose
% \tvec{x}       % vector transpose
% \V             % vocabulary
% \vec{x}        % vector
% \vq            % q (vector)
% \vx            % x (vector)
% \vy            % y (vector)
% \Z             % integer ring

% spellcheck-off

\usepackage{xparse}
\usepackage{bm,upgreek,mathtools,mathrsfs}

\DeclarePairedDelimiter\Innerp{\langle}{\rangle}
\DeclarePairedDelimiter\Floor{\lfloor}{\rfloor}
\DeclarePairedDelimiter\Ceil{\lceil}{\rceil}
\newcommand\Inner[2]{\ensuremath{\Innerp{#1, #2}}}
\DeclarePairedDelimiter\Abs{\lvert}{\rvert}
\DeclarePairedDelimiter\Card{\lvert}{\rvert}
\DeclarePairedDelimiter\Closedp{[}{]}
\newcommand\Closed[2]{\ensuremath{\Closedp{#1, #2}}}
\DeclarePairedDelimiter\Openp{]}{[}
\newcommand\Open[2]{\ensuremath{\Openp{#1, #2}}}
\DeclareRobustCommand\iff{\mathbin{\Longleftrightarrow}}
\newcommand\Four[1]{\ensuremath{\widehat{#1}}}
\DeclarePairedDelimiter\Normp{\|}{\|}
\DeclareMathOperator\supp{supp}

\ExplSyntaxOn

\DeclareDocumentCommand\IRange{m g}{%
  \ensuremath{\IfNoValueTF{#2}{1}{#1} \mathbin{:} \IfNoValueTF{#2}{#1}{#2}}%
}

\DeclareDocumentCommand\Conv{s m m O{} D(){}}{%
\ensuremath{%
  \tl_if_empty:nF {#4#5} {\left(}%
  {#2}%
  \mathbin{\IfBooleanTF{#1}{\underline{\ast}}{\ast}}%
  {#3}%
  \tl_if_empty:nF {#4} {\right)\!\left[#4\right]}%
  \tl_if_empty:nF {#5} {\right)\!\left(#5\right)}%
}%
}

\DeclareDocumentCommand\Corr{s m m O{} D(){}}{%
  \ensuremath{%
    \tl_if_empty:nF {#4#5} {\left(}%
    {#2}%
    \mathbin{\IfBooleanTF{#1}{\underline{\star}}{\star}}%
    {#3}%
    \tl_if_empty:nF {#4} {\right)\!\left[#4\right]}%
    \tl_if_empty:nF {#5} {\right)\!\left(#5\right)}%
  }%
}

\DeclareDocumentCommand\Lp{O{2} G{\mathbb{R}}}{%
  \ensuremath{\mathord{{\bm L}\textsuperscript{\ensuremath{#1}}}\left(#2\right)}%
}

\DeclareDocumentCommand\Norm{o m}{%
  \ensuremath{\Normp{#2}\IfNoValueF{#1}{\textsubscript{\ensuremath{#1}}}}%
}

\NewDocumentCommand\Vector{m}
 {
  \commexo_vector:n { #1 }
 }

\cs_new_protected:Npn \commexo_vector:n #1
 {
  \tl_map_inline:nn { #1 }
   {
    \commexo_vector_inner:n { ##1 }
   }
 }

\cs_new_protected:Npn \commexo_vector_inner:n #1
 {
  \tl_if_in:VnTF \g_commexo_latin_tl { #1 }
   {% we check whether the argument is a Latin letter
    \mathbf { #1 } % a Latin letter
   }
   {% if not a Latin letter, we check if it's an uppercase Greek letter
    \tl_if_in:VnTF \g_commexo_ucgreek_tl { #1 }
     {
      \bm { #1 } % a Greek uppercase letter
     }
     {% if not, we check if it's a lowercase Greek letter
      \tl_if_in:VnTF \g_commexo_lcgreek_tl { #1 }
       {
        \commexo_makeboldupright:n { #1 }
       }
       {% none of the above, just issue #1
        #1 % fall back
       }
     }
   }
 }

\cs_new_protected:Npn \commexo_makeboldupright:n #1
 {
  \bm { \use:c { up \cs_to_str:N #1 } }
 }

\tl_new:N \g_commexo_latin_tl
\tl_new:N \g_commexo_ucgreek_tl
\tl_new:N \g_commexo_lcgreek_tl
\tl_gset:Nn \g_commexo_latin_tl
 {
  ABCDEFGHIJKLMNOPQRSTUVWXYZ
  abcdefghijklmnopqrstuvwxyz
 }
\tl_gset:Nn \g_commexo_ucgreek_tl
 {
  \Gamma\Delta\Theta\Lambda\Pi\Sigma\Upsilon\Phi\Chi\Psi\Omega
 }
\tl_gset:Nn \g_commexo_lcgreek_tl
 {
  \alpha\beta\gamma\delta\epsilon\zeta\eta\theta\iota\kappa
  \lambda\mu\nu\xi\pi\rho\sigma\tau\upsilon\phi\chi\psi\omega
  \varepsilon\vartheta\varpi\varphi\varsigma\varrho
 }

\DeclareDocumentCommand\Pr{O{} m G{} G{}}{ %
      \tl_if_empty:nTF {#1}
        {\text{P}}
        {\text{P\textsubscript{\ensuremath{#1}}}}
      \ensuremath{
      \tl_if_empty:nTF {#3}
        {\tl_if_empty:nTF {#4}
          {\left(#2\right)}
          {\left(#2\middle\|#4\right)}}
        {\tl_if_empty:nTF {#4}
          {\left(#2\middle|#3\right)}
          {\left(#2\middle|#3\middle\|#4\right)}}
      }
  }

\DeclareDocumentCommand\Ex{O{} m}
  {
    \tl_if_empty:nTF {#1}
      {\mathbb{E}}
      {\mathbb{E\text{\textsubscript{\ensuremath{#1}}}}}
    \ensuremath{\left[#2\right]}
  }

\DeclareDocumentCommand\EnumSet{s m G{}}
  {
    \ensuremath{
      \IfBooleanTF{#1} {
        \tl_if_empty:nTF {#3}
          {\left(#2\right)}
          {\left(#2\middle|#3\right)}
      }
      {
        \tl_if_empty:nTF {#3}
          {\left\{#2\right\}}
          {\left\{#2\middle|#3\right\}}
      }
    }
  }

\DeclareDocumentCommand{\Powerset}{ G{} }
  {
    \ensuremath{
      \tl_if_empty:nTF {#1}
        {\mathscr{P}}
        {\mathscr{P}\left({#1}\right)}
    }
  }

\ExplSyntaxOff

\renewcommand\vec[1]{\ensuremath{\Vector{#1}}}
\newcommand\tvec[1]{\ensuremath{\Vector{#1}^\intercal}}
\newcommand\matr\vec
\newcommand\tmatr\tvec
\newcommand\field[1]{\ensuremath{\mathbb{#1}}}
\newcommand\ring\field
\newcommand\semi\ring
\newcommand\set[1]{\ensuremath{\mathcal{#1}}}
\newcommand\coll[1]{\ensuremath{\mathscr{#1}}}
\newcommand\Inv[1]{\ensuremath{{#1}^{\text{-}1}}}
\newcommand\Dual[1]{\ensuremath{{#1}\prime}}
\newcommand\alg[1]{\ensuremath{\mathfrak{#1}}}
\newcommand\setint[1]{\ensuremath{\int\limits_{#1}\mathrm{d}#1\>}}
\newcommand\Pdiv[2]{\ensuremath{\frac{\partial{#1}}{\partial{#2}}}}
\newcommand\Conj[1]{\ensuremath{\overline{#1}}}
\newcommand\asto{\ensuremath{\mathbin{\overset{a.s.}{\to}}}}

\newcommand\R{\ensuremath{\field{R}}}
\newcommand\C{\ensuremath{\field{C}}}
\newcommand\N{\ensuremath{\semi{N}}}
\newcommand\Z{\ring{Z}}
\newcommand\Q{\field{Q}}
\newcommand\V{\set{V}}

% 
% In document:
% 
% \Abs{x}        % absolute value
% \alg{A}        % algebra
% \C             % complex field
% \Card{x}       % cardinality
% \Ceil{x}       % ceiling
% \Closed{x}{y}  % closed interval
% \coll{A}       % collection
% \Conj{x}       % conjugate
% \Conv{x}{y}    % convolution - x * y
% \Conv{x}{y}[t] % convolution - (x * y)[t]
% \Conv*{x}{y}   % convolution (alternate) - x (* w/ underline) y
% \Corr{x}{y}    % correlation - x star y
% \Corr{x}{y}[t] % correlation - (x star y)[t]
% \Corr*{x}{y}   % correlation (alternate) - x (star w/ underline) y
% \Dual{x}       % dual
% \Ex[y]{x}      % expectation - E_y[x]
% \Ex{x}         % expectation - E[x]
% \field{A}      % field
% \Floor{x}      % floor
% \Fourier{x}    % Fourier transform of x
% x \iff y       % if-and-only-if
% x \asto y      % almost sure convergence to
% \Inner{x}{y}   % inner product
% \Inv{x}        % inverse
% \IRange{a}     % range - 1:a
% \IRange{a}{b}  % range - a:b
% \matr{x}       % matrix
% \N             % natural semiring
% \Norm[y]{x}    % norm - ||x||_y
% \Norm{x}       % norm - ||x||
% \Open{x}{y}    % open interval
% \Pdiv{x}{y}    % partial derivative
% \Pr[y]{x}      % probability - P_y(x)
% \Pr{x}         % probability - P(x)
% \Pr{x}{}{y}    % probability - P(x||y)
% \Pr{x}{y}      % probability - P(x|y)
% \Pr{x}{y}{z}   % probability - P(x|y||z)
% \Q             % rational field
% \ring{A}       % ring
% \semi{A}       % semiring
% \set{A}        % set
% \SInt{A}       % set integral
% \supp x        % support
% \tmat{x}       % matrix transpose
% \tvec{x}       % vector transpose
% \V             % vocabulary
% \vec{x}        % vector
% \vq            % q (vector)
% \vx            % x (vector)
% \vy            % y (vector)
% \Z             % integer ring

% spellcheck-off

\usepackage{xparse}
\usepackage{bm,upgreek,mathtools,mathrsfs}

\DeclarePairedDelimiter\Innerp{\langle}{\rangle}
\DeclarePairedDelimiter\Floor{\lfloor}{\rfloor}
\DeclarePairedDelimiter\Ceil{\lceil}{\rceil}
\newcommand\Inner[2]{\ensuremath{\Innerp{#1, #2}}}
\DeclarePairedDelimiter\Abs{\lvert}{\rvert}
\DeclarePairedDelimiter\Card{\lvert}{\rvert}
\DeclarePairedDelimiter\Closedp{[}{]}
\newcommand\Closed[2]{\ensuremath{\Closedp{#1, #2}}}
\DeclarePairedDelimiter\Openp{]}{[}
\newcommand\Open[2]{\ensuremath{\Openp{#1, #2}}}
\DeclareRobustCommand\iff{\mathbin{\Longleftrightarrow}}
\newcommand\Four[1]{\ensuremath{\widehat{#1}}}
\DeclarePairedDelimiter\Normp{\|}{\|}
\DeclareMathOperator\supp{supp}

\ExplSyntaxOn

\DeclareDocumentCommand\IRange{m g}{%
  \ensuremath{\IfNoValueTF{#2}{1}{#1} \mathbin{:} \IfNoValueTF{#2}{#1}{#2}}%
}

\DeclareDocumentCommand\Conv{s m m O{} D(){}}{%
\ensuremath{%
  \tl_if_empty:nF {#4#5} {\left(}%
  {#2}%
  \mathbin{\IfBooleanTF{#1}{\underline{\ast}}{\ast}}%
  {#3}%
  \tl_if_empty:nF {#4} {\right)\!\left[#4\right]}%
  \tl_if_empty:nF {#5} {\right)\!\left(#5\right)}%
}%
}

\DeclareDocumentCommand\Corr{s m m O{} D(){}}{%
  \ensuremath{%
    \tl_if_empty:nF {#4#5} {\left(}%
    {#2}%
    \mathbin{\IfBooleanTF{#1}{\underline{\star}}{\star}}%
    {#3}%
    \tl_if_empty:nF {#4} {\right)\!\left[#4\right]}%
    \tl_if_empty:nF {#5} {\right)\!\left(#5\right)}%
  }%
}

\DeclareDocumentCommand\Lp{O{2} G{\mathbb{R}}}{%
  \ensuremath{\mathord{{\bm L}\textsubscript{\ensuremath{#1}}}\left(#2\right)}%
}

\DeclareDocumentCommand\Norm{o m}{%
  \ensuremath{\Normp{#2}\IfNoValueF{#1}{\textsubscript{\ensuremath{#1}}}}%
}

\NewDocumentCommand\Vector{m}
 {
  \commexo_vector:n { #1 }
 }

\cs_new_protected:Npn \commexo_vector:n #1
 {
  \tl_map_inline:nn { #1 }
   {
    \commexo_vector_inner:n { ##1 }
   }
 }

\cs_new_protected:Npn \commexo_vector_inner:n #1
 {
  \tl_if_in:VnTF \g_commexo_latin_tl { #1 }
   {% we check whether the argument is a Latin letter
    \mathbf { #1 } % a Latin letter
   }
   {% if not a Latin letter, we check if it's an uppercase Greek letter
    \tl_if_in:VnTF \g_commexo_ucgreek_tl { #1 }
     {
      \bm { #1 } % a Greek uppercase letter
     }
     {% if not, we check if it's a lowercase Greek letter
      \tl_if_in:VnTF \g_commexo_lcgreek_tl { #1 }
       {
        \commexo_makeboldupright:n { #1 }
       }
       {% none of the above, just issue #1
        #1 % fall back
       }
     }
   }
 }

\cs_new_protected:Npn \commexo_makeboldupright:n #1
 {
  \bm { \use:c { up \cs_to_str:N #1 } }
 }

\tl_new:N \g_commexo_latin_tl
\tl_new:N \g_commexo_ucgreek_tl
\tl_new:N \g_commexo_lcgreek_tl
\tl_gset:Nn \g_commexo_latin_tl
 {
  ABCDEFGHIJKLMNOPQRSTUVWXYZ
  abcdefghijklmnopqrstuvwxyz
 }
\tl_gset:Nn \g_commexo_ucgreek_tl
 {
  \Gamma\Delta\Theta\Lambda\Pi\Sigma\Upsilon\Phi\Chi\Psi\Omega
 }
\tl_gset:Nn \g_commexo_lcgreek_tl
 {
  \alpha\beta\gamma\delta\epsilon\zeta\eta\theta\iota\kappa
  \lambda\mu\nu\xi\pi\rho\sigma\tau\upsilon\phi\chi\psi\omega
  \varepsilon\vartheta\varpi\varphi\varsigma\varrho
 }

\DeclareDocumentCommand\Pr{O{} m G{} G{}}{ %
      \tl_if_empty:nTF {#1}
        {\text{P}}
        {\text{P\textsubscript{\ensuremath{#1}}}}
      \ensuremath{
      \tl_if_empty:nTF {#3}
        {\tl_if_empty:nTF {#4}
          {\left(#2\right)}
          {\left(#2\middle\|#4\right)}}
        {\tl_if_empty:nTF {#4}
          {\left(#2\middle|#3\right)}
          {\left(#2\middle|#3\middle\|#4\right)}}
      }
  }

\DeclareDocumentCommand\Ex{O{} m}
  {
    \tl_if_empty:nTF {#1}
      {\mathbb{E}}
      {\mathbb{E\text{\textsubscript{\ensuremath{#1}}}}}
    \ensuremath{\left[#2\right]}
  }

\DeclareDocumentCommand\EnumSet{s m G{}}
  {
    \ensuremath{
      \IfBooleanTF{#1} {
        \tl_if_empty:nTF {#3}
          {\left(#2\right)}
          {\left(#2\middle|#3\right)}
      }
      {
        \tl_if_empty:nTF {#3}
          {\left\{#2\right\}}
          {\left\{#2\middle|#3\right\}}
      }
    }
  }

\DeclareDocumentCommand{\Powerset}{ G{} }
  {
    \ensuremath{
      \tl_if_empty:nTF {#1}
        {\mathscr{P}}
        {\mathscr{P}\left({#1}\right)}
    }
  }

\ExplSyntaxOff

\renewcommand\vec[1]{\ensuremath{\Vector{#1}}}
\newcommand\tvec[1]{\ensuremath{\Vector{#1}^\intercal}}
\newcommand\matr\vec
\newcommand\tmatr\tvec
\newcommand\field[1]{\ensuremath{\mathbb{#1}}}
\newcommand\ring\field
\newcommand\semi\ring
\newcommand\set[1]{\ensuremath{\mathcal{#1}}}
\newcommand\coll[1]{\ensuremath{\mathscr{#1}}}
\newcommand\Inv[1]{\ensuremath{{#1}^{\text{-}1}}}
\newcommand\Dual[1]{\ensuremath{{#1}\prime}}
\newcommand\alg[1]{\ensuremath{\mathfrak{#1}}}
\newcommand\setint[1]{\ensuremath{\int\limits_{#1}\mathrm{d}#1\>}}
\newcommand\Pdiv[2]{\ensuremath{\frac{\partial{#1}}{\partial{#2}}}}
\newcommand\Conj[1]{\ensuremath{\overline{#1}}}
\newcommand\asto{\ensuremath{\mathbin{\overset{a.s.}{\to}}}}

\newcommand\R{\ensuremath{\field{R}}}
\newcommand\C{\ensuremath{\field{C}}}
\newcommand\N{\ensuremath{\semi{N}}}
\newcommand\Z{\ring{Z}}
\newcommand\Q{\field{Q}}
\newcommand\V{\set{V}}
